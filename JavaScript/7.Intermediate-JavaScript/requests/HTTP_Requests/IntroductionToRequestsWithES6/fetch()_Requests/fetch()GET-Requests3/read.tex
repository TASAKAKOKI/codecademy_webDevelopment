fetch() GET Requests III
    In the previous exercise, you wrote the boilerplate code for a GET request using fetch() and .then(). In this exercise, you’re going to use that code and manipulate it to access the Datamuse API and render information in the browser.

        Datamuse API(https://www.datamuse.com/api/)
        
    If the request is successful, you’ll get back an array of words that sound like the word you typed into the input field.

    You may get some errors as you complete each step. This is because sometimes we’ve split a single step into one or more steps to make it easier to follow. By the end, you should be receiving no errors.