fetch() GET Requests IV
    Great job making it this far!

    In the previous exercise, you created the query URL, called the fetch() function and passed it the query URL and a settings object. Then, you chained a .then() method and passed it two functions as arguments — one to handle the promise if it resolves, and one to handle network errors if the promise is rejected.

    In this exercise, you’ll now take the information that was returned with the promise and manipulate the webpage!