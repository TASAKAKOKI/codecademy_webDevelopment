Writing async Functions
    We’ve seen that the await keyword halts the execution of an async function until a promise is no longer pending. Don’t forget the await keyword! It may seem obvious, but this can be a tricky mistake to catch because our function will still run— it just won’t have the desired results.

    We’re going to explore this using the following function, which returns a promise that resolves to 'Yay, I resolved!' after a 1 second delay:
        let myPromise = () => {
        return new Promise((resolve, reject) => {
            setTimeout(() => {
            resolve('Yay, I resolved!')
            }, 1000);
        });
        }
    
    Now we’ll write two async functions which invoke myPromise():
        async function noAwait() {
        let value = myPromise();
        console.log(value);
        }

        async function yesAwait() {
        let value = await myPromise();
        console.log(value);
        }

        noAwait(); // Prints: Promise { <pending> }
        yesAwait(); // Prints: Yay, I resolved!

    In the first async function, noAwait(), we left off the await keyword before myPromise(). In the second, yesAwait(), we included it. The noAwait() function logs Promise { <pending> } to the console. Without the await keyword, the function execution wasn’t paused. The console.log() on the following line was executed before the promise had resolved.

    Remember that the await operator returns the resolved value of a promise. When used properly in yesAwait(), the variable value was assigned the resolved value of the myPromise() promise, whereas in noAwait(), value was assigned the promise object itself.