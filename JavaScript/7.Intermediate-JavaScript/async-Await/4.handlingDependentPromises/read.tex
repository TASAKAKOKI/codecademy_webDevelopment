Handling Dependent Promises
    The true beauty of async...await is when we have a series of asynchronous actions which depend on one another. For example, we may make a network request based on a query to a database. In that case, we would need to wait to make the network request until we had the results from the database. With native promise syntax, we use a chain of .then() functions making sure to return correctly each one:

        function nativePromiseVersion() {
            returnsFirstPromise()
            .then((firstValue) => {
                console.log(firstValue);
                return returnsSecondPromise(firstValue);
            })
        .then((secondValue) => {
                console.log(secondValue);
            });
        }
    Let’s break down what’s happening in the nativePromiseVersion() function:
        - Within our function we use two functions which return promises: returnsFirstPromise() and returnsSecondPromise().
        - We invoke returnsFirstPromise() and ensure that the first promise resolved by using .then().
        - In the callback of our first .then(), we log the resolved value of the first promise, firstValue, and then return returnsSecondPromise(firstValue).
        - We use another .then() to print the second promise’s resolved value to the console.
    
    Here’s how we’d write an async function to accomplish the same thing:
        async function asyncAwaitVersion() {
        let firstValue = await returnsFirstPromise();
        console.log(firstValue);
        let secondValue = await returnsSecondPromise(firstValue);
        console.log(secondValue);
        }
    Let’s break down what’s happening in our asyncAwaitVersion() function:
        - We mark our function as async.
        - Inside our function, we create a variable firstValue assigned await returnsFirstPromise(). This means firstValue is assigned the resolved value of the awaited promise.
        - Next, we log firstValue to the console.
        - Then, we create a variable secondValue assigned to await returnsSecondPromise(firstValue). Therefore, secondValue is assigned this promise’s resolved value.
        - Finally, we log secondValue to the console.

    Though using the async...await syntax can save us some typing, the length reduction isn’t the main point.
    ***The function, the async...await version more closely resembles synchronous code, which helps developers maintain and debug their code.
    ***The async...await syntax also makes it easy to store and refer to resolved values from promises further back in our chain which is a much more difficult task with native promise syntax.